% Generated by Sphinx.
\def\sphinxdocclass{report}
\documentclass[letterpaper,10pt,english]{sphinxmanual}
\usepackage[utf8]{inputenc}
\DeclareUnicodeCharacter{00A0}{\nobreakspace}
\usepackage[T1]{fontenc}
\usepackage{babel}
\usepackage{times}
\usepackage[Bjarne]{fncychap}
\usepackage{longtable}
\usepackage{sphinx}
\usepackage{multirow}


\title{SuperHub Documentation}
\date{febrero 20, 2014}
\release{}
\author{Javier Bejar}
\newcommand{\sphinxlogo}{}
\renewcommand{\releasename}{Release}
\makeindex

\makeatletter
\def\PYG@reset{\let\PYG@it=\relax \let\PYG@bf=\relax%
    \let\PYG@ul=\relax \let\PYG@tc=\relax%
    \let\PYG@bc=\relax \let\PYG@ff=\relax}
\def\PYG@tok#1{\csname PYG@tok@#1\endcsname}
\def\PYG@toks#1+{\ifx\relax#1\empty\else%
    \PYG@tok{#1}\expandafter\PYG@toks\fi}
\def\PYG@do#1{\PYG@bc{\PYG@tc{\PYG@ul{%
    \PYG@it{\PYG@bf{\PYG@ff{#1}}}}}}}
\def\PYG#1#2{\PYG@reset\PYG@toks#1+\relax+\PYG@do{#2}}

\def\PYG@tok@gd{\def\PYG@tc##1{\textcolor[rgb]{0.63,0.00,0.00}{##1}}}
\def\PYG@tok@gu{\let\PYG@bf=\textbf\def\PYG@tc##1{\textcolor[rgb]{0.50,0.00,0.50}{##1}}}
\def\PYG@tok@gt{\def\PYG@tc##1{\textcolor[rgb]{0.00,0.25,0.82}{##1}}}
\def\PYG@tok@gs{\let\PYG@bf=\textbf}
\def\PYG@tok@gr{\def\PYG@tc##1{\textcolor[rgb]{1.00,0.00,0.00}{##1}}}
\def\PYG@tok@cm{\let\PYG@it=\textit\def\PYG@tc##1{\textcolor[rgb]{0.25,0.50,0.56}{##1}}}
\def\PYG@tok@vg{\def\PYG@tc##1{\textcolor[rgb]{0.73,0.38,0.84}{##1}}}
\def\PYG@tok@m{\def\PYG@tc##1{\textcolor[rgb]{0.13,0.50,0.31}{##1}}}
\def\PYG@tok@mh{\def\PYG@tc##1{\textcolor[rgb]{0.13,0.50,0.31}{##1}}}
\def\PYG@tok@cs{\def\PYG@tc##1{\textcolor[rgb]{0.25,0.50,0.56}{##1}}\def\PYG@bc##1{\colorbox[rgb]{1.00,0.94,0.94}{##1}}}
\def\PYG@tok@ge{\let\PYG@it=\textit}
\def\PYG@tok@vc{\def\PYG@tc##1{\textcolor[rgb]{0.73,0.38,0.84}{##1}}}
\def\PYG@tok@il{\def\PYG@tc##1{\textcolor[rgb]{0.13,0.50,0.31}{##1}}}
\def\PYG@tok@go{\def\PYG@tc##1{\textcolor[rgb]{0.19,0.19,0.19}{##1}}}
\def\PYG@tok@cp{\def\PYG@tc##1{\textcolor[rgb]{0.00,0.44,0.13}{##1}}}
\def\PYG@tok@gi{\def\PYG@tc##1{\textcolor[rgb]{0.00,0.63,0.00}{##1}}}
\def\PYG@tok@gh{\let\PYG@bf=\textbf\def\PYG@tc##1{\textcolor[rgb]{0.00,0.00,0.50}{##1}}}
\def\PYG@tok@ni{\let\PYG@bf=\textbf\def\PYG@tc##1{\textcolor[rgb]{0.84,0.33,0.22}{##1}}}
\def\PYG@tok@nl{\let\PYG@bf=\textbf\def\PYG@tc##1{\textcolor[rgb]{0.00,0.13,0.44}{##1}}}
\def\PYG@tok@nn{\let\PYG@bf=\textbf\def\PYG@tc##1{\textcolor[rgb]{0.05,0.52,0.71}{##1}}}
\def\PYG@tok@no{\def\PYG@tc##1{\textcolor[rgb]{0.38,0.68,0.84}{##1}}}
\def\PYG@tok@na{\def\PYG@tc##1{\textcolor[rgb]{0.25,0.44,0.63}{##1}}}
\def\PYG@tok@nb{\def\PYG@tc##1{\textcolor[rgb]{0.00,0.44,0.13}{##1}}}
\def\PYG@tok@nc{\let\PYG@bf=\textbf\def\PYG@tc##1{\textcolor[rgb]{0.05,0.52,0.71}{##1}}}
\def\PYG@tok@nd{\let\PYG@bf=\textbf\def\PYG@tc##1{\textcolor[rgb]{0.33,0.33,0.33}{##1}}}
\def\PYG@tok@ne{\def\PYG@tc##1{\textcolor[rgb]{0.00,0.44,0.13}{##1}}}
\def\PYG@tok@nf{\def\PYG@tc##1{\textcolor[rgb]{0.02,0.16,0.49}{##1}}}
\def\PYG@tok@si{\let\PYG@it=\textit\def\PYG@tc##1{\textcolor[rgb]{0.44,0.63,0.82}{##1}}}
\def\PYG@tok@s2{\def\PYG@tc##1{\textcolor[rgb]{0.25,0.44,0.63}{##1}}}
\def\PYG@tok@vi{\def\PYG@tc##1{\textcolor[rgb]{0.73,0.38,0.84}{##1}}}
\def\PYG@tok@nt{\let\PYG@bf=\textbf\def\PYG@tc##1{\textcolor[rgb]{0.02,0.16,0.45}{##1}}}
\def\PYG@tok@nv{\def\PYG@tc##1{\textcolor[rgb]{0.73,0.38,0.84}{##1}}}
\def\PYG@tok@s1{\def\PYG@tc##1{\textcolor[rgb]{0.25,0.44,0.63}{##1}}}
\def\PYG@tok@gp{\let\PYG@bf=\textbf\def\PYG@tc##1{\textcolor[rgb]{0.78,0.36,0.04}{##1}}}
\def\PYG@tok@sh{\def\PYG@tc##1{\textcolor[rgb]{0.25,0.44,0.63}{##1}}}
\def\PYG@tok@ow{\let\PYG@bf=\textbf\def\PYG@tc##1{\textcolor[rgb]{0.00,0.44,0.13}{##1}}}
\def\PYG@tok@sx{\def\PYG@tc##1{\textcolor[rgb]{0.78,0.36,0.04}{##1}}}
\def\PYG@tok@bp{\def\PYG@tc##1{\textcolor[rgb]{0.00,0.44,0.13}{##1}}}
\def\PYG@tok@c1{\let\PYG@it=\textit\def\PYG@tc##1{\textcolor[rgb]{0.25,0.50,0.56}{##1}}}
\def\PYG@tok@kc{\let\PYG@bf=\textbf\def\PYG@tc##1{\textcolor[rgb]{0.00,0.44,0.13}{##1}}}
\def\PYG@tok@c{\let\PYG@it=\textit\def\PYG@tc##1{\textcolor[rgb]{0.25,0.50,0.56}{##1}}}
\def\PYG@tok@mf{\def\PYG@tc##1{\textcolor[rgb]{0.13,0.50,0.31}{##1}}}
\def\PYG@tok@err{\def\PYG@bc##1{\fcolorbox[rgb]{1.00,0.00,0.00}{1,1,1}{##1}}}
\def\PYG@tok@kd{\let\PYG@bf=\textbf\def\PYG@tc##1{\textcolor[rgb]{0.00,0.44,0.13}{##1}}}
\def\PYG@tok@ss{\def\PYG@tc##1{\textcolor[rgb]{0.32,0.47,0.09}{##1}}}
\def\PYG@tok@sr{\def\PYG@tc##1{\textcolor[rgb]{0.14,0.33,0.53}{##1}}}
\def\PYG@tok@mo{\def\PYG@tc##1{\textcolor[rgb]{0.13,0.50,0.31}{##1}}}
\def\PYG@tok@mi{\def\PYG@tc##1{\textcolor[rgb]{0.13,0.50,0.31}{##1}}}
\def\PYG@tok@kn{\let\PYG@bf=\textbf\def\PYG@tc##1{\textcolor[rgb]{0.00,0.44,0.13}{##1}}}
\def\PYG@tok@o{\def\PYG@tc##1{\textcolor[rgb]{0.40,0.40,0.40}{##1}}}
\def\PYG@tok@kr{\let\PYG@bf=\textbf\def\PYG@tc##1{\textcolor[rgb]{0.00,0.44,0.13}{##1}}}
\def\PYG@tok@s{\def\PYG@tc##1{\textcolor[rgb]{0.25,0.44,0.63}{##1}}}
\def\PYG@tok@kp{\def\PYG@tc##1{\textcolor[rgb]{0.00,0.44,0.13}{##1}}}
\def\PYG@tok@w{\def\PYG@tc##1{\textcolor[rgb]{0.73,0.73,0.73}{##1}}}
\def\PYG@tok@kt{\def\PYG@tc##1{\textcolor[rgb]{0.56,0.13,0.00}{##1}}}
\def\PYG@tok@sc{\def\PYG@tc##1{\textcolor[rgb]{0.25,0.44,0.63}{##1}}}
\def\PYG@tok@sb{\def\PYG@tc##1{\textcolor[rgb]{0.25,0.44,0.63}{##1}}}
\def\PYG@tok@k{\let\PYG@bf=\textbf\def\PYG@tc##1{\textcolor[rgb]{0.00,0.44,0.13}{##1}}}
\def\PYG@tok@se{\let\PYG@bf=\textbf\def\PYG@tc##1{\textcolor[rgb]{0.25,0.44,0.63}{##1}}}
\def\PYG@tok@sd{\let\PYG@it=\textit\def\PYG@tc##1{\textcolor[rgb]{0.25,0.44,0.63}{##1}}}

\def\PYGZbs{\char`\\}
\def\PYGZus{\char`\_}
\def\PYGZob{\char`\{}
\def\PYGZcb{\char`\}}
\def\PYGZca{\char`\^}
\def\PYGZsh{\char`\#}
\def\PYGZpc{\char`\%}
\def\PYGZdl{\char`\$}
\def\PYGZti{\char`\~}
% for compatibility with earlier versions
\def\PYGZat{@}
\def\PYGZlb{[}
\def\PYGZrb{]}
\makeatother

\begin{document}

\maketitle
\tableofcontents
\phantomsection\label{index::doc}


Contents:


\chapter{\texttt{Constants} Module}
\label{index:constants-module}\label{index:welcome-to-superhub-s-documentation}\label{index:module-SuperHub.Constants}\index{SuperHub.Constants (module)}\phantomsection\label{index:module-SuperHubConstants}\index{SuperHubConstants (module)}

\section{Constants}
\label{index:constants}\begin{quote}\begin{description}
\item[{Description}] \leavevmode
SuperHub constants,

The coordinates of the region of interest and the path to the data files
And the information of the mongo database

\item[{Authors}] \leavevmode
bejar

\item[{Version}] \leavevmode
1.0

\end{description}\end{quote}


\chapter{\texttt{DB} Module}
\label{index:module-SuperHub.DB}\label{index:db-module}\index{SuperHub.DB (module)}\phantomsection\label{index:module-DB}\index{DB (module)}

\section{DB}
\label{index:db}\begin{quote}\begin{description}
\item[{Description}] \leavevmode
SuperHub data functions

Exports data from database to csv file

Loads data from csv file

Performs different processings to the data matrix

\item[{Authors}] \leavevmode
bejar

\item[{Version}] \leavevmode
1.0

\end{description}\end{quote}
\index{getApplicationData() (in module SuperHub.DB)}

\begin{fulllineitems}
\phantomsection\label{index:SuperHub.DB.getApplicationData}\pysiglinewithargsret{\code{SuperHub.DB.}\bfcode{getApplicationData}}{\emph{application}}{}
Get the data events from the database and saves it in a csv file
\begin{quote}\begin{description}
\item[{Param }] \leavevmode
application

\item[{Param }] \leavevmode
cpath

\item[{Param }] \leavevmode
square

\end{description}\end{quote}

\end{fulllineitems}

\index{getApplicationData2() (in module SuperHub.DB)}

\begin{fulllineitems}
\phantomsection\label{index:SuperHub.DB.getApplicationData2}\pysiglinewithargsret{\code{SuperHub.DB.}\bfcode{getApplicationData2}}{}{}
\end{fulllineitems}

\index{getApplicationDataOne() (in module SuperHub.DB)}

\begin{fulllineitems}
\phantomsection\label{index:SuperHub.DB.getApplicationDataOne}\pysiglinewithargsret{\code{SuperHub.DB.}\bfcode{getApplicationDataOne}}{\emph{application}}{}~\begin{quote}\begin{description}
\item[{Param }] \leavevmode
application:

\end{description}\end{quote}

\end{fulllineitems}

\index{getLApplicationData() (in module SuperHub.DB)}

\begin{fulllineitems}
\phantomsection\label{index:SuperHub.DB.getLApplicationData}\pysiglinewithargsret{\code{SuperHub.DB.}\bfcode{getLApplicationData}}{\emph{lapplication}}{}
Retrieves data from a lists of Social applications
Saves an individual file for each application
and a file with all the data
\begin{quote}\begin{description}
\item[{Parameters}] \leavevmode
\textbf{lapplication} -- 

\end{description}\end{quote}

\end{fulllineitems}

\index{saveDataResult() (in module SuperHub.DB)}

\begin{fulllineitems}
\phantomsection\label{index:SuperHub.DB.saveDataResult}\pysiglinewithargsret{\code{SuperHub.DB.}\bfcode{saveDataResult}}{\emph{data}, \emph{fname}}{}
Save data results in a file
\begin{quote}\begin{description}
\item[{Param }] \leavevmode
data:

\item[{Param }] \leavevmode
fname:

\end{description}\end{quote}

\end{fulllineitems}

\index{transferApplicationData() (in module SuperHub.DB)}

\begin{fulllineitems}
\phantomsection\label{index:SuperHub.DB.transferApplicationData}\pysiglinewithargsret{\code{SuperHub.DB.}\bfcode{transferApplicationData}}{\emph{application}}{}
Trasfers data from
\begin{quote}\begin{description}
\item[{Param }] \leavevmode
application:

\end{description}\end{quote}

\end{fulllineitems}



\chapter{\texttt{Data} Module}
\label{index:data-module}\label{index:module-SuperHub.Data}\index{SuperHub.Data (module)}\phantomsection\label{index:module-Data}\index{Data (module)}

\section{Data}
\label{index:data}\begin{quote}\begin{description}
\item[{Description}] \leavevmode
SuperHub Data class

Performs different processings to the data matrix

\item[{Authors}] \leavevmode
bejar

\item[{Version}] \leavevmode
1.0

\end{description}\end{quote}

File: Data

Created on 18/02/2014 10:09

@author: bejar
\index{Data (class in SuperHub.Data)}

\begin{fulllineitems}
\phantomsection\label{index:SuperHub.Data.Data}\pysiglinewithargsret{\strong{class }\code{SuperHub.Data.}\bfcode{Data}}{\emph{path}, \emph{application}}{}
Class for a superhub dataset

dataset = numpy array
application = Name of the data file
cpath = Path of the data file
mxhh = maximum position of the heavy hitters list
mnhh = minimum position of the heavy hitters list
lhh = list of users ordered by the number of elements in the dataset
\index{application (SuperHub.Data.Data attribute)}

\begin{fulllineitems}
\phantomsection\label{index:SuperHub.Data.Data.application}\pysigline{\bfcode{application}\strong{ = None}}
\end{fulllineitems}

\index{compute\_heavy\_hitters() (SuperHub.Data.Data method)}

\begin{fulllineitems}
\phantomsection\label{index:SuperHub.Data.Data.compute_heavy_hitters}\pysiglinewithargsret{\bfcode{compute\_heavy\_hitters}}{\emph{mxhh}, \emph{mnhh}}{}
Computes the list of the number of events
and returns a list with the users between the
positions mxhh and mnhh in the descendent order

If the list heavy hitters have already been computed it is reused
\begin{quote}\begin{description}
\item[{Param }] \leavevmode
data:

\item[{Param }] \leavevmode
mxhh:

\item[{Param }] \leavevmode
mnhh:

\item[{Returns}] \leavevmode
list with the list of users

\end{description}\end{quote}

\end{fulllineitems}

\index{contingency() (SuperHub.Data.Data method)}

\begin{fulllineitems}
\phantomsection\label{index:SuperHub.Data.Data.contingency}\pysiglinewithargsret{\bfcode{contingency}}{\emph{scale}, \emph{distrib=True}}{}
Generates an scale x scale accumulated plot of the events
\begin{quote}\begin{description}
\item[{Param }] \leavevmode
data:

\item[{Param }] \leavevmode
scale:

\item[{Param }] \leavevmode
distrib:

\end{description}\end{quote}

\end{fulllineitems}

\index{daily\_table() (SuperHub.Data.Data method)}

\begin{fulllineitems}
\phantomsection\label{index:SuperHub.Data.Data.daily_table}\pysiglinewithargsret{\bfcode{daily\_table}}{}{}
Computes the accumulated events by day for the data table
\begin{quote}\begin{description}
\item[{Returns}] \leavevmode
A daily table

\end{description}\end{quote}

\end{fulllineitems}

\index{dataset (SuperHub.Data.Data attribute)}

\begin{fulllineitems}
\phantomsection\label{index:SuperHub.Data.Data.dataset}\pysigline{\bfcode{dataset}\strong{ = None}}
\end{fulllineitems}

\index{datasethh (SuperHub.Data.Data attribute)}

\begin{fulllineitems}
\phantomsection\label{index:SuperHub.Data.Data.datasethh}\pysigline{\bfcode{datasethh}\strong{ = None}}
\end{fulllineitems}

\index{get\_dataset() (SuperHub.Data.Data method)}

\begin{fulllineitems}
\phantomsection\label{index:SuperHub.Data.Data.get_dataset}\pysiglinewithargsret{\bfcode{get\_dataset}}{}{}
Returns the numpy array that represents the dataset
@return:

\end{fulllineitems}

\index{hourly\_table() (SuperHub.Data.Data method)}

\begin{fulllineitems}
\phantomsection\label{index:SuperHub.Data.Data.hourly_table}\pysiglinewithargsret{\bfcode{hourly\_table}}{}{}
Computes the accumulated events by hour for the data table
\begin{quote}\begin{description}
\item[{Returns}] \leavevmode
An hourly table

\end{description}\end{quote}

\end{fulllineitems}

\index{lhh (SuperHub.Data.Data attribute)}

\begin{fulllineitems}
\phantomsection\label{index:SuperHub.Data.Data.lhh}\pysigline{\bfcode{lhh}\strong{ = None}}
\end{fulllineitems}

\index{mnhh (SuperHub.Data.Data attribute)}

\begin{fulllineitems}
\phantomsection\label{index:SuperHub.Data.Data.mnhh}\pysigline{\bfcode{mnhh}\strong{ = None}}
\end{fulllineitems}

\index{monthly\_table() (SuperHub.Data.Data method)}

\begin{fulllineitems}
\phantomsection\label{index:SuperHub.Data.Data.monthly_table}\pysiglinewithargsret{\bfcode{monthly\_table}}{}{}
Computes the accumulated events by month

@return: A montly rable

\end{fulllineitems}

\index{mxhh (SuperHub.Data.Data attribute)}

\begin{fulllineitems}
\phantomsection\label{index:SuperHub.Data.Data.mxhh}\pysigline{\bfcode{mxhh}\strong{ = None}}
\end{fulllineitems}

\index{read\_data() (SuperHub.Data.Data method)}

\begin{fulllineitems}
\phantomsection\label{index:SuperHub.Data.Data.read_data}\pysiglinewithargsret{\bfcode{read\_data}}{}{}
Loads the data from the csv file
\begin{quote}\begin{description}
\item[{Param }] \leavevmode
application:

\item[{Returns}] \leavevmode


\end{description}\end{quote}

\end{fulllineitems}

\index{select\_data\_users() (SuperHub.Data.Data method)}

\begin{fulllineitems}
\phantomsection\label{index:SuperHub.Data.Data.select_data_users}\pysiglinewithargsret{\bfcode{select\_data\_users}}{\emph{users}}{}
Selects only the events from the list of users
Returns a new object with the selected users
\begin{quote}\begin{description}
\item[{Param }] \leavevmode
users: List of users to select

\item[{Returns}] \leavevmode


\end{description}\end{quote}

\end{fulllineitems}

\index{select\_heavy\_hitters() (SuperHub.Data.Data method)}

\begin{fulllineitems}
\phantomsection\label{index:SuperHub.Data.Data.select_heavy_hitters}\pysiglinewithargsret{\bfcode{select\_heavy\_hitters}}{\emph{mxhh}, \emph{mnhh}}{}
Deletes all the events that are not from the heavy hitters
Returns a new data object only with the heavy hitters

@param mxhh:
@param mnhh:
@return: A list of the most active users in the indicated range

\end{fulllineitems}

\index{wpath (SuperHub.Data.Data attribute)}

\begin{fulllineitems}
\phantomsection\label{index:SuperHub.Data.Data.wpath}\pysigline{\bfcode{wpath}\strong{ = None}}
\end{fulllineitems}


\end{fulllineitems}



\chapter{\texttt{Descriptive} Module}
\label{index:module-SuperHub.Descriptive}\label{index:descriptive-module}\index{SuperHub.Descriptive (module)}\phantomsection\label{index:module-Descriptive}\index{Descriptive (module)}

\section{Descriptive}
\label{index:descriptive}\begin{quote}\begin{description}
\item[{Description}] \leavevmode
SuperHub Descriptive data functions

\end{description}\end{quote}

Functions for computing descriptive statistics from the dataset

For now mainly histograms
\begin{quote}\begin{description}
\item[{Authors}] \leavevmode
bejar

\item[{Version}] \leavevmode
1.0

\end{description}\end{quote}

File: Descriptive

Created on 20/02/2014 15:23

@author: bejar
\index{data\_histograms() (in module SuperHub.Descriptive)}

\begin{fulllineitems}
\phantomsection\label{index:SuperHub.Descriptive.data_histograms}\pysiglinewithargsret{\code{SuperHub.Descriptive.}\bfcode{data\_histograms}}{\emph{application}, \emph{lhh=None}}{}
Generate histograms for different characteristics of the data
Outputs the data used to generate the histograms
\begin{itemize}
\item {} 
Number of daily events

\item {} 
Number of days of users

\item {} 
Accumulated events per hour

\item {} 
Accumulated ecents per weekday

\end{itemize}
\begin{quote}\begin{description}
\item[{Param }] \leavevmode
application:

\item[{Param }] \leavevmode
lhh:

\end{description}\end{quote}

\end{fulllineitems}

\index{plot\_accumulated\_events() (in module SuperHub.Descriptive)}

\begin{fulllineitems}
\phantomsection\label{index:SuperHub.Descriptive.plot_accumulated_events}\pysiglinewithargsret{\code{SuperHub.Descriptive.}\bfcode{plot\_accumulated\_events}}{\emph{data}, \emph{distrib=True}, \emph{scale=100}}{}
Plots the accumulated geographical events in the selected area to the
specified scale
\begin{quote}\begin{description}
\item[{Param }] \leavevmode
application: name of the data file

\item[{Param }] \leavevmode
distrib: whether the PDF or the absolute numbers are plotted

\item[{Param }] \leavevmode
scale: scale of the discretization

\end{description}\end{quote}

\end{fulllineitems}

\index{user\_events\_histogram() (in module SuperHub.Descriptive)}

\begin{fulllineitems}
\phantomsection\label{index:SuperHub.Descriptive.user_events_histogram}\pysiglinewithargsret{\code{SuperHub.Descriptive.}\bfcode{user\_events\_histogram}}{\emph{data}, \emph{scale=100}, \emph{timeres=4}}{}
Histogram of the number of places-time a user has been
\begin{quote}\begin{description}
\item[{Param }] \leavevmode
scale:

\item[{Param }] \leavevmode
application:

\item[{Param }] \leavevmode
mxhh:

\item[{Param }] \leavevmode
mnhh:

\end{description}\end{quote}

\end{fulllineitems}



\chapter{\texttt{Plots} Module}
\label{index:plots-module}\label{index:module-SuperHub.Plots}\index{SuperHub.Plots (module)}\phantomsection\label{index:module-SuperHub.Plot}\index{SuperHub.Plot (module)}

\section{Plot}
\label{index:plot}\begin{quote}\begin{description}
\item[{Description}] \leavevmode
Different plots of the data

\item[{Authors}] \leavevmode
bejar

\item[{Version}] \leavevmode
1.0

\end{description}\end{quote}
\index{daily\_histogram() (in module SuperHub.Plots)}

\begin{fulllineitems}
\phantomsection\label{index:SuperHub.Plots.daily_histogram}\pysiglinewithargsret{\code{SuperHub.Plots.}\bfcode{daily\_histogram}}{\emph{data}}{}
Plot of events accumulated by week day
\begin{quote}\begin{description}
\item[{Param }] \leavevmode
application:

\item[{Param }] \leavevmode
mxhh:

\item[{Param }] \leavevmode
mnhh:

\end{description}\end{quote}

\end{fulllineitems}

\index{hourly\_histogram() (in module SuperHub.Plots)}

\begin{fulllineitems}
\phantomsection\label{index:SuperHub.Plots.hourly_histogram}\pysiglinewithargsret{\code{SuperHub.Plots.}\bfcode{hourly\_histogram}}{\emph{data}}{}
Plots of events accumulated by hours

\end{fulllineitems}

\index{montly\_histogram() (in module SuperHub.Plots)}

\begin{fulllineitems}
\phantomsection\label{index:SuperHub.Plots.montly_histogram}\pysiglinewithargsret{\code{SuperHub.Plots.}\bfcode{montly\_histogram}}{\emph{data}}{}
Plots the events accumulated by month

@param application:
@param mxhh:
@param mnhh:
@return:

\end{fulllineitems}

\index{plotHisto() (in module SuperHub.Plots)}

\begin{fulllineitems}
\phantomsection\label{index:SuperHub.Plots.plotHisto}\pysiglinewithargsret{\code{SuperHub.Plots.}\bfcode{plotHisto}}{\emph{data}, \emph{bins}}{}
Plots a histogram
\begin{quote}\begin{description}
\item[{Param }] \leavevmode
data:

\item[{Param }] \leavevmode
bins:

\end{description}\end{quote}

\end{fulllineitems}

\index{saveHisto() (in module SuperHub.Plots)}

\begin{fulllineitems}
\phantomsection\label{index:SuperHub.Plots.saveHisto}\pysiglinewithargsret{\code{SuperHub.Plots.}\bfcode{saveHisto}}{\emph{data}, \emph{bins}, \emph{fname}}{}
Saves a histogram
\begin{quote}\begin{description}
\item[{Param }] \leavevmode
data:

\item[{Param }] \leavevmode
bins:

\item[{Param }] \leavevmode
fname:

\end{description}\end{quote}

\end{fulllineitems}

\index{savePlot() (in module SuperHub.Plots)}

\begin{fulllineitems}
\phantomsection\label{index:SuperHub.Plots.savePlot}\pysiglinewithargsret{\code{SuperHub.Plots.}\bfcode{savePlot}}{\emph{axis}, \emph{data}, \emph{fname}}{}
Saves a plot of the data using the values of axis
\begin{quote}\begin{description}
\item[{Param }] \leavevmode
data:

\item[{Param }] \leavevmode
num:

\item[{Param }] \leavevmode
fname:

\end{description}\end{quote}

\end{fulllineitems}



\chapter{\texttt{Routes} Module}
\label{index:routes-module}\label{index:module-SuperHub.Routes}\index{SuperHub.Routes (module)}\phantomsection\label{index:module-Routes}\index{Routes (module)}

\section{Routes}
\label{index:routes}\begin{quote}\begin{description}
\item[{Description}] \leavevmode
Routes

Routines that compute routes

\item[{Authors}] \leavevmode
bejar

\item[{Version}] \leavevmode
1.0

\end{description}\end{quote}

File: Routes

Created on 20/02/2014 15:17

@author: bejar
\index{transaction\_routes() (in module SuperHub.Routes)}

\begin{fulllineitems}
\phantomsection\label{index:SuperHub.Routes.transaction_routes}\pysiglinewithargsret{\code{SuperHub.Routes.}\bfcode{transaction\_routes}}{\emph{data}, \emph{nfile}, \emph{scale=100}, \emph{supp=30}, \emph{timeres=4.0}, \emph{colapsed=False}}{}
Generates a diagram of the routes obtained by the frequent itemsets fp-growth algorithm
\begin{quote}\begin{description}
\item[{Param }] \leavevmode
dataclean:

\item[{Param }] \leavevmode
application:

\item[{Param }] \leavevmode
mxhh:

\item[{Param }] \leavevmode
mnhh:

\item[{Param }] \leavevmode
scale:

\item[{Param }] \leavevmode
supp:

\item[{Param }] \leavevmode
timeres:

\end{description}\end{quote}

\end{fulllineitems}

\index{transaction\_routes\_many() (in module SuperHub.Routes)}

\begin{fulllineitems}
\phantomsection\label{index:SuperHub.Routes.transaction_routes_many}\pysiglinewithargsret{\code{SuperHub.Routes.}\bfcode{transaction\_routes\_many}}{\emph{data}, \emph{lhh=None}, \emph{lscale=None}, \emph{supp=30}, \emph{ltimeres=None}, \emph{colapsed=False}}{}
Computes the diagrams of frequent routes for a list of parameters
\begin{quote}\begin{description}
\item[{Param }] \leavevmode
application:

\item[{Param }] \leavevmode
lhh:

\item[{Param }] \leavevmode
lscale:

\item[{Param }] \leavevmode
supp:

\item[{Param }] \leavevmode
ltimeres:

\end{description}\end{quote}

\end{fulllineitems}



\chapter{\texttt{Transactions} Module}
\label{index:module-SuperHub.Transactions}\label{index:transactions-module}\index{SuperHub.Transactions (module)}\phantomsection\label{index:module-Transactions}\index{Transactions (module)}

\section{Transactions}
\label{index:transactions}\begin{quote}\begin{description}
\item[{Description}] \leavevmode
Transactions,

Class for transactions processing

\item[{Authors}] \leavevmode
bejar

\item[{Version}] \leavevmode
1.0

\end{description}\end{quote}

Created on 18/02/2014 10:59

@author: bejar
\index{DailyDiscretizedTransactions (class in SuperHub.Transactions)}

\begin{fulllineitems}
\phantomsection\label{index:SuperHub.Transactions.DailyDiscretizedTransactions}\pysiglinewithargsret{\strong{class }\code{SuperHub.Transactions.}\bfcode{DailyDiscretizedTransactions}}{\emph{data}, \emph{scale=100}, \emph{timeres=4.0}}{}
Bases: {\hyperref[index:SuperHub.Transactions.DailyTransactions]{\code{SuperHub.Transactions.DailyTransactions}}}

Class for the daily discretized transactions

\end{fulllineitems}

\index{DailyTransactions (class in SuperHub.Transactions)}

\begin{fulllineitems}
\phantomsection\label{index:SuperHub.Transactions.DailyTransactions}\pysiglinewithargsret{\strong{class }\code{SuperHub.Transactions.}\bfcode{DailyTransactions}}{\emph{data}}{}
Bases: {\hyperref[index:SuperHub.Transactions.Transactions]{\code{SuperHub.Transactions.Transactions}}}

Class for the daily transactions
\index{colapse() (SuperHub.Transactions.DailyTransactions method)}

\begin{fulllineitems}
\phantomsection\label{index:SuperHub.Transactions.DailyTransactions.colapse}\pysiglinewithargsret{\bfcode{colapse}}{}{}
Colapses the transactions of a user on a set with all the different items
in the transactions (basically where has been and when (considering the
discretization used) during the period of time covered by the transactions
\begin{quote}\begin{description}
\item[{Param }] \leavevmode
trans: Dictionary of user/time transactions

\item[{Returns}] \leavevmode
Dictionary of daily transactions

\end{description}\end{quote}

\end{fulllineitems}

\index{colapse\_count() (SuperHub.Transactions.DailyTransactions method)}

\begin{fulllineitems}
\phantomsection\label{index:SuperHub.Transactions.DailyTransactions.colapse_count}\pysiglinewithargsret{\bfcode{colapse\_count}}{}{}
Colapsed the transactions of a user on a dictionary with all the different items in the
transctions, counting how many times the user has been at that time at that place (considering
the discretization used)
@return:

\end{fulllineitems}

\index{save() (SuperHub.Transactions.DailyTransactions method)}

\begin{fulllineitems}
\phantomsection\label{index:SuperHub.Transactions.DailyTransactions.save}\pysiglinewithargsret{\bfcode{save}}{\emph{rfile}}{}
Saves the daily transactions in a file
\begin{quote}\begin{description}
\item[{Param }] \leavevmode
nfile:

\item[{Param }] \leavevmode
application:

\item[{Param }] \leavevmode
mxhh:

\item[{Param }] \leavevmode
mnhh:

\item[{Param }] \leavevmode
scale:

\end{description}\end{quote}

\end{fulllineitems}

\index{serialize() (SuperHub.Transactions.DailyTransactions method)}

\begin{fulllineitems}
\phantomsection\label{index:SuperHub.Transactions.DailyTransactions.serialize}\pysiglinewithargsret{\bfcode{serialize}}{}{}
Transforms the transactions from dictionaries to lists
\begin{quote}\begin{description}
\item[{Param }] \leavevmode
trans:

\item[{Returns}] \leavevmode


\end{description}\end{quote}

\end{fulllineitems}

\index{users\_daily\_length() (SuperHub.Transactions.DailyTransactions method)}

\begin{fulllineitems}
\phantomsection\label{index:SuperHub.Transactions.DailyTransactions.users_daily_length}\pysiglinewithargsret{\bfcode{users\_daily\_length}}{}{}
Computes the list of lengths of the daily transactions for all users

\end{fulllineitems}

\index{users\_prevalence() (SuperHub.Transactions.DailyTransactions method)}

\begin{fulllineitems}
\phantomsection\label{index:SuperHub.Transactions.DailyTransactions.users_prevalence}\pysiglinewithargsret{\bfcode{users\_prevalence}}{}{}
Computes the number of daily transactions for all users

\end{fulllineitems}


\end{fulllineitems}

\index{Transactions (class in SuperHub.Transactions)}

\begin{fulllineitems}
\phantomsection\label{index:SuperHub.Transactions.Transactions}\pysiglinewithargsret{\strong{class }\code{SuperHub.Transactions.}\bfcode{Transactions}}{\emph{data}}{}
Class for the user transactions
\index{application (SuperHub.Transactions.Transactions attribute)}

\begin{fulllineitems}
\phantomsection\label{index:SuperHub.Transactions.Transactions.application}\pysigline{\bfcode{application}\strong{ = None}}
\end{fulllineitems}

\index{usertrans (SuperHub.Transactions.Transactions attribute)}

\begin{fulllineitems}
\phantomsection\label{index:SuperHub.Transactions.Transactions.usertrans}\pysigline{\bfcode{usertrans}\strong{ = None}}
\end{fulllineitems}

\index{wpath (SuperHub.Transactions.Transactions attribute)}

\begin{fulllineitems}
\phantomsection\label{index:SuperHub.Transactions.Transactions.wpath}\pysigline{\bfcode{wpath}\strong{ = None}}
\end{fulllineitems}


\end{fulllineitems}



\chapter{\texttt{Util} Module}
\label{index:util-module}\label{index:module-SuperHub.Util}\index{SuperHub.Util (module)}\phantomsection\label{index:module-Util}\index{Util (module)}

\section{Util}
\label{index:util}\begin{quote}\begin{description}
\item[{Description}] \leavevmode
Util

Different Auxiliary functions used for different purposes

\item[{Authors}] \leavevmode
bejar

\item[{Version}] \leavevmode
1.0

\end{description}\end{quote}

File: Util

Created on 20/02/2014 14:12

@author: bejar
\index{diff\_items() (in module SuperHub.Util)}

\begin{fulllineitems}
\phantomsection\label{index:SuperHub.Util.diff_items}\pysiglinewithargsret{\code{SuperHub.Util.}\bfcode{diff\_items}}{\emph{seq}}{}
Number of different geo point in a sequence
\begin{quote}\begin{description}
\item[{Param }] \leavevmode
seq:

\item[{Returns}] \leavevmode


\end{description}\end{quote}

\end{fulllineitems}

\index{item\_key\_sort() (in module SuperHub.Util)}

\begin{fulllineitems}
\phantomsection\label{index:SuperHub.Util.item_key_sort}\pysiglinewithargsret{\code{SuperHub.Util.}\bfcode{item\_key\_sort}}{\emph{v}}{}
auxiliary function for sorting geo-time events
\begin{quote}\begin{description}
\item[{Param }] \leavevmode
v:

\item[{Returns}] \leavevmode


\end{description}\end{quote}

\end{fulllineitems}



\chapter{Indices and tables}
\label{index:indices-and-tables}\begin{itemize}
\item {} 
\emph{genindex}

\item {} 
\emph{modindex}

\item {} 
\emph{search}

\end{itemize}


\renewcommand{\indexname}{Python Module Index}
\begin{theindex}
\def\bigletter#1{{\Large\sffamily#1}\nopagebreak\vspace{1mm}}
\bigletter{d}
\item {\texttt{Data}}, \pageref{index:module-Data}
\item {\texttt{DB}}, \pageref{index:module-DB}
\item {\texttt{Descriptive}}, \pageref{index:module-Descriptive}
\indexspace
\bigletter{r}
\item {\texttt{Routes}}, \pageref{index:module-Routes}
\indexspace
\bigletter{s}
\item {\texttt{SuperHub.Constants}}, \pageref{index:module-SuperHub.Constants}
\item {\texttt{SuperHub.Data}}, \pageref{index:module-SuperHub.Data}
\item {\texttt{SuperHub.DB}}, \pageref{index:module-SuperHub.DB}
\item {\texttt{SuperHub.Descriptive}}, \pageref{index:module-SuperHub.Descriptive}
\item {\texttt{SuperHub.Plot}}, \pageref{index:module-SuperHub.Plot}
\item {\texttt{SuperHub.Plots}}, \pageref{index:module-SuperHub.Plots}
\item {\texttt{SuperHub.Routes}}, \pageref{index:module-SuperHub.Routes}
\item {\texttt{SuperHub.Transactions}}, \pageref{index:module-SuperHub.Transactions}
\item {\texttt{SuperHub.Util}}, \pageref{index:module-SuperHub.Util}
\item {\texttt{SuperHubConstants}}, \pageref{index:module-SuperHubConstants}
\indexspace
\bigletter{t}
\item {\texttt{Transactions}}, \pageref{index:module-Transactions}
\indexspace
\bigletter{u}
\item {\texttt{Util}}, \pageref{index:module-Util}
\end{theindex}

\renewcommand{\indexname}{Index}
\printindex
\end{document}
